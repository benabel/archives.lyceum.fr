% Transistor CMOS-N
\documentclass{article}
\usepackage{tikz}
%%%<
\usepackage{verbatim}
\usepackage[active,tightpage]{preview}
\usepackage{fontspec}
\setmainfont{TeX Gyre Adventor}
\PreviewEnvironment{tikzpicture}
\setlength\PreviewBorder{10pt}%
%%%>
\usepackage[european]{circuitikz}
\usepackage{amsmath}
\begin{document}
\newfontfamily\titlefont[Numbers={Proportional,OldStyle}]{Ubuntu}
\newfontfamily\normalfont[Numbers={Proportional,OldStyle}]{TeX Gyre Adventor}


%%----------------------------------------------------------------------
%%  Transistor N-mos
%%----------------------------------------------------------------------
\begin{circuitikz}[thick, line cap=round]
	% Symbol with voltage and current:
	\draw
	(-0.5,2) node{\titlefont\textbf{Transistor N-mos}}
	(0,0) node[nmos] (mos) {}
	(mos.gate) node[anchor=east] {\normalfont Grille}
	(mos.drain) node[anchor=south] {Drain}
	(mos.source) node[anchor=north] {Source}
	;
\end{circuitikz}

%%----------------------------------------------------------------------
%%  Commutation du transistor
%%----------------------------------------------------------------------
\begin{circuitikz}[scale=1.0, thick]
	\draw (1.5,3) node{\titlefont\textbf{Commutation du transistor}};

	\draw (-1,0) -- (0,0) node[anchor=north west] {Source} to [short, -o] (0,2/3);
	\draw (0,4/3) to [short,o-](0,2) node[anchor=south west] {Drain};
	\draw (0-0.01,2/3+0.05) -- +(96:2/3);
	\draw (0,0) -- (-1,0);
	\draw (-1,1) to [short, -o] (-1,1) node[anchor=south] {\normalfont Grille};
	\draw [->](-1,0.1) -- (-1,0.9) node[anchor=north east] {$U>U_{seuil}$};
	\draw (-1,-1) node{\titlefont{Transistor saturé}};

	\draw (1.5,-1) -- (1.5,2.5);


	\draw (4,0) -- (5,0) node[anchor=north west] {Source} to [short, -o] (5,2/3);
	\draw (5,4/3) to [short,o-](5,2) node[anchor=south west] {Drain};
	\draw (5-0.03, 2/3+0.05) -- +(120:2/3);
	\draw (5,0) -- (4,0);
	\draw (4,1) to [short, -o] (4,1) node[anchor=south] {\normalfont Grille};
	\draw [->](4,0.1) -- (4,0.9) node[anchor=north east] {$U<U_{seuil}$};
	\draw (4,-1) node{\titlefont{Transistor bloqué}};

\end{circuitikz}


%%----------------------------------------------------------------------
%%  Le circuit Non
%%----------------------------------------------------------------------

\begin{circuitikz} [thick, line cap=round]
	% Symbol with voltage and current:
	\draw
	(2,4) node{\titlefont\textbf{Le circuit Non}}

	(0,0) node[nmos] (mos) {}
	(mos.gate) node[anchor=south] {\normalfont G}

	(mos.drain) node[anchor=north west] {D}
	(mos.drain) to [R] (0, 3) -- (-3,3)
	(mos.source) node[anchor=south west] {S}
	-- (0,-1) -- (-3,-1)
	to[V=$E$] (-3,3)
	(-1,-2) node{\titlefont{Schéma électrique}};

	\draw[red]
	(mos.gate) to [short, -o]  +(-0.5,0)  node[anchor=east] {\normalfont\textit{entrée}}
	;
	\draw[red]
	(mos.drain) to [short, -o] +(+0.5,0)  node[anchor=west] {\normalfont\textit{sortie}}
	;

	\draw (2,-1) -- (2,3);

	\draw
	(6,1) node[not port] (notgate) {}
	(5,-2) node{\titlefont{Schéma logique}};

	\draw[red]
	(notgate.in) to [short, -o]  +(-0.5,0)  node[anchor=east] {\normalfont\textit{entrée}}
	;
	\draw[red]
	(notgate.out) to [short, -o] +(+0.5,0)  node[anchor=west] {\normalfont\textit{sortie}}

	;
\end{circuitikz}

%%----------------------------------------------------------------------
%%  Le circuit Ou
%%----------------------------------------------------------------------

\begin{circuitikz} [thick, line cap=round]
	% Symbol with voltage and current:
	\draw
	(1,5) node{\titlefont\textbf{Le circuit Ou}}

	(0,0) node[nmos] (mos) {}
	(3,0) node[nmos] (mos2) {}
	(5,1) node[nmos] (mos3) {}


	(mos.source) -- (0,-1) -- (-3,-1) to[V=$E$] (-3,4)
	(mos.drain) -- (0,1)
	(mos2.source) -- (3,-1)
	(mos2.drain) -- (3,1)
	(mos3.source) -- (5,-1) -- (0,-1)
	(mos3.gate) -- (0,1) to [R] (0, 4) -- (-3,4)
	(mos3.drain)  to [R] (5, 4) -- (0,4)

	(1,-2) node{\titlefont{Schéma électrique}}

	;


	\draw[red]
	(mos.gate) to [short, -o]  +(-0.5,0)  node[anchor=east] {\normalfont\textit{entrée 1}}
	;
	\draw[red]
	(mos2.gate) to [short, -o]  +(-0.5,0)  node[anchor=east] {\normalfont\textit{entrée 2}}
	;
	\draw[red]
	(mos3.drain)  to [short, -o] +(+0.5,0)  node[anchor=west] {\normalfont\textit{sortie}}
	;

	\draw (-4,-3) -- (7,-3);

	\draw
	(2,-5) node[or port] (orgate) {}
	(1,-6.5) node{\titlefont{Schéma logique}};

	\draw[red]
	(orgate.in 1) to [short, -o]  +(-0.5,0)  node[anchor=east] {\normalfont\textit{entrée 1}};
	\draw[red]
	(orgate.in 2) to [short, -o]  +(-0.5,0)  node[anchor=east] {\normalfont\textit{entrée 2}}
	;
	\draw[red]
	(orgate.out) to [short, -o] +(+0.5,0)  node[anchor=west] {\normalfont\textit{sortie}}

	;
\end{circuitikz}

%%----------------------------------------------------------------------
%%  Le circuit Et
%%----------------------------------------------------------------------

\begin{circuitikz} [thick, line cap=round]
	% Symbol with voltage and current:
	\draw
	(0,5) node{\titlefont\textbf{Le circuit Et}}

	(0,1) node[nmos] (mos) {}
	(0,0) node[nmos] (mos2) {}
	(3,1.5) node[nmos] (mos3) {}


	(mos2.source) -- (0,-1) -- (-4,-1) to[V=$E$] (-4,4)
	(mos3.gate) -- (0,1.5) to [R] (0, 4) -- (-4,4)
	(mos3.drain)  to [R] (3, 4) -- (0,4)
	(mos3.source)  -- (3, -1) -- (0,-1)

	(0,-2) node{\titlefont{Schéma électrique}}

	;


	\draw[red]
	(mos.gate) to [short, -o]  +(-0.5,0)  node[anchor=east] {\normalfont\textit{entrée 1}}
	;
	\draw[red]
	(mos2.gate) to [short, -o]  +(-0.5,0)  node[anchor=east] {\normalfont\textit{entrée 2}}
	;
	\draw[red]
	(mos3.drain)  to [short, -o] +(+0.5,0)  node[anchor=west] {\normalfont\textit{sortie}}
	;

	\draw (-5,-3) -- (5,-3);

	\draw
	(1,-5) node[and port] (andgate) {}
	(0,-6.5) node{\titlefont{Schéma logique}};

	\draw[red]
	(andgate.in 1) to [short, -o]  +(-0.5,0)  node[anchor=east] {\normalfont\textit{entrée 1}};
	\draw[red]
	(andgate.in 2) to [short, -o]  +(-0.5,0)  node[anchor=east] {\normalfont\textit{entrée 2}}
	;
	\draw[red]
	(andgate.out) to [short, -o] +(+0.5,0)  node[anchor=west] {\normalfont\textit{sortie}}

	;
\end{circuitikz}


\end{document}
